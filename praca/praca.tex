\documentclass[pl]{minipw} % wszystkie ustawienia szablonu są w minipw.cls; if in English, change [pl] to [en]
\allowdisplaybreaks
\usepackage{indentfirst}
%\usepackage[hidelinks]{hyperref}
\usepackage[all]{nowidow}
\usepackage{caption}
\usepackage{graphicx}
\usepackage{tabularx}
\usepackage{polski}
\usepackage{bm}
\usepackage{amsfonts}
\usepackage{amsmath}
\usepackage[utf8]{inputenc}
\usepackage{indentfirst}
\usepackage{float}
\usepackage{enumitem}
\usepackage{listings}
\setlength{\parindent}{5mm} % wcięcie akapitowe 5mm, zarządzenie Rektora
\usepackage{array}

\begin{document}
\sloppy


% 4. Spis treści
\tableofcontents

% 5. Treść

\cleardoublepage
\pagestyle{fancy}

\chapter*{Wprowadzenie}

\chapter{Klasyfikatory}

\chapter{Metryki}

\chapter{Benchmarki}
Wszystkie z poniższych zestawów danych dotyczą przyznawania kredytów na podstawie informacji o osobie ubiegającej się.
\section{Japanese Credit Screening Data Set}
Dane zostały utworzone na podstawie rozmów z przedstawicielami japońskiej firmy przyznającej kredyty. Kolumny nie zostały nigdzie opisane. \\
\textbf{Kolumny:} 6 liczbowych, 9 zawierających napisy \\
\textbf{Liczba wierszy:} 690 \\
\textbf{Rodzaj zdania:} Klasyfikacja \\
\textbf{Zmienna objaśniana:} + lub -, czyli przyznanie lub nieprzyznanie kredytu \\
\textbf{Rozkład zmiennej objaśnionej:} klasa "+"\ występuje w 45\% obserwacji \\
\begin{table}[H]
\caption{Pierwsze wiersze danych}
\label{use_case_tab}
\centering
\begin{tabular}{|c|c|c|c|c|c|c|c|c|c|c|c|c|c|c|c|}
\hline

b  & 30.83  & 0  & u  & g  & w  & v  & 1.25  & t  & t  & 1  & f  & g  & 00202  & 0  & +\\
a  & 58.67  & 4.46  & u  & g  & q  & h  & 3.04  & t  & t  & 6  & f  & g  & 00043  & 560  & +\\
a  & 24.50  & 0.5  & u  & g  & q  & h  & 1.5  & t  & f  & 0  & f  & g  & 00280  & 824  & +\\
b  & 27.83  & 1.54  & u  & g  & w  & v  & 3.75  & t  & t  & 5  & t  & g  & 00100  & 3  & +\\
b  & 20.17  & 5.625  & u  & g  & w  & v  & 1.71  & t  & f  & 0  & f  & s  & 00120  & 0  & +\\
\hline
\end{tabular}
\end{table}

\section{German Credit Data}
Poniższe dane zostały dostarczone przez profesora Hansa Hofmanna z Uniwersytetu w Hamburgu. Udostępnione zostały dwa pliki: jeden zawierający kolumny jakościowe i ilościowe i drugi, zawierający tylko kolumny ilościowe. Ze względu na łatwość interpretacji dalej wykorzystywany będzie pierwszy zestaw.
Wraz z danymi został dostarczony ich dokładny opis, zawierający nazwy kolumn i znaczenie wartości w poszczególnych kolumnach. \\
\textbf{Kolumny:} 7 kolumn ilościowych i 13 jakościowych \\
\textbf{Liczba wierszy:} 1000 \\
\textbf{Rodzaj zdania:} Klasyfikacja \\
\textbf{Zmienna objaśniana:} 1 - przyznanie lub 2 - nieprzyznanie kredytu \\
\textbf{Rozkład zmiennej objaśnionej:} klasa "2"\ występuje w 30\% obserwacji \\

\newcolumntype{L}{>{\centering\arraybackslash}m{#1}}

\begin{table}[H]
\caption{Znaczenie kolumn}
\begin{tabular}{|l|L|L|}
\hline
& Nazwa atrybutu & Przyjmowane wartości \\
\hline
1 & Status istniejących kont & brak konta; 0 DM; do 200 DM; powyżej 200 DM \\
\hline
2 & Czas trwania pożyczki & wartość numeryczna w miesiącach \\
\hline
3 & Historia kredytowa & brak historii kredytowej; wszystkie kredyty spłacone; opóźnienia w spłacie; konto z niespłaconymi kredytami \\
\hline
4 & Cel pożyczki & nowy samochód; używany samochód; meble / wyposażenie domu; radio / telewizor; urządzenia domowe; naprawy; edukacja; wakacje; szkolenia zawodowe; pożyczka biznesowa; inne \\
\hline
5 & Wielkość pożyczki & wartość numeryczna w DM \\
\hline
6 & Stan konta oszczędnościowego lub obligacyjnego & brak takich kont; do 100 DM; do 500 DM; do 1000 DM; co najmniej 1000 DM \\
\hline
7 & Długość trwania aktualnego stosunku pracy & bezrobotny; do jednego roku; do 4 lat; do 7 lat; co najmniej 7 lat \\
\hline
8 & Rata pożyczki jako procent rozporządzalnego dochodu & wartość numeryczna w procentach \\
\hline
9 & Stan cywilny oraz płeć & mężczyzna rozwiedziony lub w separacji; kobieta rozwiedziona, w separacji lub zamężna; wolny mężczyzna; mężczyzna zamężny lub owdowiały; wolna kobieta \\
\hline
10 & Inni wierzyciele/wnioskodawcy & brak; wnioskodawca; żyrant      \\
\hline
11 & Czas zamieszkania w aktualnym miejscu & wartość numeryczna: rok \\
\hline
12 & Posiadana własność & brak; nieruchomość; wkład do spółdzielni mieszkaniowej / ubezpieczenie na życie; samochód \\
\hline
13 & Wiek & wartość numeryczna w latach \\
\hline
14 & Inne posiadane plany ratalne & brak; bankowy; w sieci sklepów \\
\hline
15 & Stan posiadanie domu & wynajmowany; posiadany na własność; za darmo \\
\hline
16 & Liczba kredytów posiadanych w rozpatrywanym banku & wartość numeryczna \\
\hline
17 & Praca & bezrobotny/niewykwalifikowany (nie stały mieszkaniec); niewykwalifikowany (stały mieszkaniec); pracownik wykwalifikowany / urzędnik; kadra zarządcza / samozatrudniony / pracownik wysoce wykwalifikowany \\
\hline
18 & Liczba osób zależnych finansowo od wnioskodawcy & wartość numeryczna \\
\hline
19 & Telefon & brak; zarejestrowany na nazwisko wnioskodawcy \\
\hline
20 & Pracownik zagraniczny & tak; nie \\
\hline
\end{tabular}
\end{table}

\begin{table}[H]
\caption{Pierwsze wiersze danych}
\label{use_case_tab}
\centering
\scalebox{0.7}{
\begin{tabular}{|c|c|c|c|c|c|c|c|c|c|c|c|c|c|c|c|c|c|c|c|c|}
\hline
1 & 2 & 3 & 4 & 5 & 6 & 7 & 8 & 9 & 10 & 11 & 12 & 13 & 14 & 15 & 16 & 17 & 18 & 19 & 20 & 21 \\
\hline
A11  & 6  & A34  & A43  & 1169  & A65  & A75  & 4  & A93  & A101  & 4  & A121  & 67  & A143  & A152  & 2  & A173  & 1  & A192  & A201  & 1\\
A12  & 48  & A32  & A43  & 5951  & A61  & A73  & 2  & A92  & A101  & 2  & A121  & 22  & A143  & A152  & 1  & A173  & 1  & A191  & A201  & 2\\
A14  & 12  & A34  & A46  & 2096  & A61  & A74  & 2  & A93  & A101  & 3  & A121  & 49  & A143  & A152  & 1  & A172  & 2  & A191  & A201  & 1\\
A11  & 42  & A32  & A42  & 7882  & A61  & A74  & 2  & A93  & A103  & 4  & A122  & 45  & A143  & A153  & 1  & A173  & 2  & A191  & A201  & 1\\
A11  & 24  & A33  & A40  & 4870  & A61  & A73  & 3  & A93  & A101  & 4  & A124  & 53  & A143  & A153  & 2  & A173  & 2  & A191  & A201  & 2\\
\hline
\end{tabular}
}
\end{table}

Opis danych zawiera proponowaną funkcję jakości klasyfikacji, zakładającą większą karę za niesłuszne przyznanie kredytu, niż za niesłuszne nieprzyznanie.

\section{Australian Credit Approval}
Dane dotyczą przyznawania kart kredytowych. \\
\textbf{Kolumny:} 6 kolumn ilościowych i 8 jakościowych \\
\textbf{Liczba wierszy:} 690 \\
\textbf{Rodzaj zdania:} Klasyfikacja \\
\textbf{Zmienna objaśniana:} 0 - nieprzyznanie lub 1 - przyznanie karty \\
\textbf{Rozkład zmiennej objaśnionej:} klasa "1"\ występuje w 56\% obserwacji \\
 
\begin{table}[H]
\caption{Pierwsze wiersze danych}
\label{use_case_tab}
\centering
\begin{tabular}{|c|c|c|c|c|c|c|c|c|c|c|c|c|c|c|c|}
\hline
1  & 22.08  & 11.46  & 2  & 4  & 4  & 1.585  & 0  & 0  & 0  & 1  & 2  & 100  & 1213  & 0\\
0  & 22.67  & 7  & 2  & 8  & 4  & 0.165  & 0  & 0  & 0  & 0  & 2  & 160  & 1  & 0\\
0  & 29.58  & 1.75  & 1  & 4  & 4  & 1.25  & 0  & 0  & 0  & 1  & 2  & 280  & 1  & 0\\
0  & 21.67  & 11.5  & 1  & 5  & 3  & 0  & 1  & 1  & 11  & 1  & 2  & 0  & 1  & 1\\
1  & 20.17  & 8.17  & 2  & 6  & 4  & 1.96  & 1  & 1  & 14  & 0  & 2  & 60  & 159  & 1\\
\hline
\end{tabular}
\end{table}

\section{Give Me Some Credit}

\textbf{Kolumny:} 6 kolumn ilościowych i 8 jakościowych \\
\textbf{Liczba wierszy:} 690 \\
\textbf{Rodzaj zdania:} Klasyfikacja \\
\textbf{Zmienna objaśniana:} 0 - nieprzyznanie lub 1 - przyznanie karty \\
\textbf{Rozkład zmiennej objaśnionej:} klasa "1"\ występuje w 56\% obserwacji \\

% 6. Bibliografia
% Bibliografia leksykograficznie wg nazwisk autorów

\begin{thebibliography}{20}%jak ktoś ma więcej książek, to niech wpisze większą liczbę
% \bibitem[numerek]{referencja} Autor, \emph{Tytuł}, Wydawnictwo, rok, strony
% cytowanie: \cite{referencja1, referencja 2,...}

\end{thebibliography}



% 7. Wykaz symboli i skrótów - jeśli nie ma, zakomentować
\chapter*{Wykaz symboli i skrótów}

\begin{tabularx}{\textwidth}{cX}
\textbf{Benchmark} & test wydajności i~jakości treningu \\

\end{tabularx}


% 8. Spis rysunków - jeśli nie ma, zakomentować (ale być może po prostu się nie zrobi)
\listoffigures


% 9. Spis tabel - jak wyżej
\listoftables

\chapter*{Dodatek. Instrukcja obsługi aplikacji}
\subsection*{Przygotowanie}
 
\subsection*{Uruchomienie aplikacji}

\clearpage

\end{document}
