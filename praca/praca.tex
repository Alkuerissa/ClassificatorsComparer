\documentclass[pl]{minipw} % wszystkie ustawienia szablonu są w minipw.cls; if in English, change [pl] to [en]
\allowdisplaybreaks
\usepackage{indentfirst}
%\usepackage[hidelinks]{hyperref}
\usepackage[all]{nowidow}
\usepackage{caption}
\usepackage{graphicx}
\usepackage{tabularx}
\usepackage{polski}
\usepackage{bm}
\usepackage{amsfonts}
\usepackage{amsmath}
\usepackage[utf8]{inputenc}
\usepackage{indentfirst}
\usepackage{float}
\usepackage{enumitem}
\usepackage{listings}
\setlength{\parindent}{5mm} % wcięcie akapitowe 5mm, zarządzenie Rektora


\begin{document}
\sloppy


% 4. Spis treści
\tableofcontents

% 5. Treść

\cleardoublepage
\pagestyle{fancy}

\chapter*{Wprowadzenie}

\chapter{Klasyfikatory}

\chapter{Metryki}

\chapter{Benchmarki}
\section{Japanese Credit Screening Data Set}
Dane dotyczą przyznawania kredytów na podstawie informacji o osobie ubiegającej się. Zostały zebrane na podstawie rozmów z przedstawicielami japońskiej firmy przyznającej kredyty. Dane składają się z 16 kolumn: 6 liczbowych, 9 zawierających napisy i jednej binarnej -- objaśniającej. Kolumny nie zostały nigdzie opisane.

\begin{table}[H]
\caption{Opisy przypadków użycia}
\label{use_case_tab}
\centering
\begin{tabular}{|c|c|c|c|c|c|c|c|c|c|c|c|c|c|c|c|}
\hline

b  & 30.83  & 0  & u  & g  & w  & v  & 1.25  & t  & t  & 1  & f  & g  & 00202  & 0  & +\\
a  & 58.67  & 4.46  & u  & g  & q  & h  & 3.04  & t  & t  & 6  & f  & g  & 00043  & 560  & +\\
a  & 24.50  & 0.5  & u  & g  & q  & h  & 1.5  & t  & f  & 0  & f  & g  & 00280  & 824  & +\\
b  & 27.83  & 1.54  & u  & g  & w  & v  & 3.75  & t  & t  & 5  & t  & g  & 00100  & 3  & +\\
b  & 20.17  & 5.625  & u  & g  & w  & v  & 1.71  & t  & f  & 0  & f  & s  & 00120  & 0  & +\\
\hline
\end{tabular}
\end{table}

% 6. Bibliografia
% Bibliografia leksykograficznie wg nazwisk autorów

\begin{thebibliography}{20}%jak ktoś ma więcej książek, to niech wpisze większą liczbę
% \bibitem[numerek]{referencja} Autor, \emph{Tytuł}, Wydawnictwo, rok, strony
% cytowanie: \cite{referencja1, referencja 2,...}

\end{thebibliography}



% 7. Wykaz symboli i skrótów - jeśli nie ma, zakomentować
\chapter*{Wykaz symboli i skrótów}

\begin{tabularx}{\textwidth}{cX}
\textbf{Benchmark} & test wydajności i~jakości treningu \\

\end{tabularx}


% 8. Spis rysunków - jeśli nie ma, zakomentować (ale być może po prostu się nie zrobi)
\listoffigures


% 9. Spis tabel - jak wyżej
\listoftables

\chapter*{Dodatek. Instrukcja obsługi aplikacji}
\subsection*{Przygotowanie}
 
\subsection*{Uruchomienie aplikacji}

\clearpage

\end{document}